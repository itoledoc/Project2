% Generated by Sphinx.
\def\sphinxdocclass{article}
\documentclass[a4paper,10pt,english]{sphinxhowto}
\usepackage[utf8]{inputenc}
\DeclareUnicodeCharacter{00A0}{\nobreakspace}
\usepackage{cmap}
\usepackage[T1]{fontenc}
\usepackage{babel}
\usepackage{times}
\usepackage[Bjarne]{fncychap}
\usepackage{longtable}
\usepackage{sphinx}
\usepackage{multirow}


\title{OpenStreetMap data wrangling with MongoDb Documentation}
\date{March 10, 2015}
\release{}
\author{Ignacio Toledo}
\newcommand{\sphinxlogo}{}
\renewcommand{\releasename}{Release}
\makeindex

\makeatletter
\def\PYG@reset{\let\PYG@it=\relax \let\PYG@bf=\relax%
    \let\PYG@ul=\relax \let\PYG@tc=\relax%
    \let\PYG@bc=\relax \let\PYG@ff=\relax}
\def\PYG@tok#1{\csname PYG@tok@#1\endcsname}
\def\PYG@toks#1+{\ifx\relax#1\empty\else%
    \PYG@tok{#1}\expandafter\PYG@toks\fi}
\def\PYG@do#1{\PYG@bc{\PYG@tc{\PYG@ul{%
    \PYG@it{\PYG@bf{\PYG@ff{#1}}}}}}}
\def\PYG#1#2{\PYG@reset\PYG@toks#1+\relax+\PYG@do{#2}}

\expandafter\def\csname PYG@tok@gd\endcsname{\def\PYG@tc##1{\textcolor[rgb]{0.63,0.00,0.00}{##1}}}
\expandafter\def\csname PYG@tok@gu\endcsname{\let\PYG@bf=\textbf\def\PYG@tc##1{\textcolor[rgb]{0.50,0.00,0.50}{##1}}}
\expandafter\def\csname PYG@tok@gt\endcsname{\def\PYG@tc##1{\textcolor[rgb]{0.00,0.27,0.87}{##1}}}
\expandafter\def\csname PYG@tok@gs\endcsname{\let\PYG@bf=\textbf}
\expandafter\def\csname PYG@tok@gr\endcsname{\def\PYG@tc##1{\textcolor[rgb]{1.00,0.00,0.00}{##1}}}
\expandafter\def\csname PYG@tok@cm\endcsname{\let\PYG@it=\textit\def\PYG@tc##1{\textcolor[rgb]{0.25,0.50,0.56}{##1}}}
\expandafter\def\csname PYG@tok@vg\endcsname{\def\PYG@tc##1{\textcolor[rgb]{0.73,0.38,0.84}{##1}}}
\expandafter\def\csname PYG@tok@m\endcsname{\def\PYG@tc##1{\textcolor[rgb]{0.13,0.50,0.31}{##1}}}
\expandafter\def\csname PYG@tok@mh\endcsname{\def\PYG@tc##1{\textcolor[rgb]{0.13,0.50,0.31}{##1}}}
\expandafter\def\csname PYG@tok@cs\endcsname{\def\PYG@tc##1{\textcolor[rgb]{0.25,0.50,0.56}{##1}}\def\PYG@bc##1{\setlength{\fboxsep}{0pt}\colorbox[rgb]{1.00,0.94,0.94}{\strut ##1}}}
\expandafter\def\csname PYG@tok@ge\endcsname{\let\PYG@it=\textit}
\expandafter\def\csname PYG@tok@vc\endcsname{\def\PYG@tc##1{\textcolor[rgb]{0.73,0.38,0.84}{##1}}}
\expandafter\def\csname PYG@tok@il\endcsname{\def\PYG@tc##1{\textcolor[rgb]{0.13,0.50,0.31}{##1}}}
\expandafter\def\csname PYG@tok@go\endcsname{\def\PYG@tc##1{\textcolor[rgb]{0.20,0.20,0.20}{##1}}}
\expandafter\def\csname PYG@tok@cp\endcsname{\def\PYG@tc##1{\textcolor[rgb]{0.00,0.44,0.13}{##1}}}
\expandafter\def\csname PYG@tok@gi\endcsname{\def\PYG@tc##1{\textcolor[rgb]{0.00,0.63,0.00}{##1}}}
\expandafter\def\csname PYG@tok@gh\endcsname{\let\PYG@bf=\textbf\def\PYG@tc##1{\textcolor[rgb]{0.00,0.00,0.50}{##1}}}
\expandafter\def\csname PYG@tok@ni\endcsname{\let\PYG@bf=\textbf\def\PYG@tc##1{\textcolor[rgb]{0.84,0.33,0.22}{##1}}}
\expandafter\def\csname PYG@tok@nl\endcsname{\let\PYG@bf=\textbf\def\PYG@tc##1{\textcolor[rgb]{0.00,0.13,0.44}{##1}}}
\expandafter\def\csname PYG@tok@nn\endcsname{\let\PYG@bf=\textbf\def\PYG@tc##1{\textcolor[rgb]{0.05,0.52,0.71}{##1}}}
\expandafter\def\csname PYG@tok@no\endcsname{\def\PYG@tc##1{\textcolor[rgb]{0.38,0.68,0.84}{##1}}}
\expandafter\def\csname PYG@tok@na\endcsname{\def\PYG@tc##1{\textcolor[rgb]{0.25,0.44,0.63}{##1}}}
\expandafter\def\csname PYG@tok@nb\endcsname{\def\PYG@tc##1{\textcolor[rgb]{0.00,0.44,0.13}{##1}}}
\expandafter\def\csname PYG@tok@nc\endcsname{\let\PYG@bf=\textbf\def\PYG@tc##1{\textcolor[rgb]{0.05,0.52,0.71}{##1}}}
\expandafter\def\csname PYG@tok@nd\endcsname{\let\PYG@bf=\textbf\def\PYG@tc##1{\textcolor[rgb]{0.33,0.33,0.33}{##1}}}
\expandafter\def\csname PYG@tok@ne\endcsname{\def\PYG@tc##1{\textcolor[rgb]{0.00,0.44,0.13}{##1}}}
\expandafter\def\csname PYG@tok@nf\endcsname{\def\PYG@tc##1{\textcolor[rgb]{0.02,0.16,0.49}{##1}}}
\expandafter\def\csname PYG@tok@si\endcsname{\let\PYG@it=\textit\def\PYG@tc##1{\textcolor[rgb]{0.44,0.63,0.82}{##1}}}
\expandafter\def\csname PYG@tok@s2\endcsname{\def\PYG@tc##1{\textcolor[rgb]{0.25,0.44,0.63}{##1}}}
\expandafter\def\csname PYG@tok@vi\endcsname{\def\PYG@tc##1{\textcolor[rgb]{0.73,0.38,0.84}{##1}}}
\expandafter\def\csname PYG@tok@nt\endcsname{\let\PYG@bf=\textbf\def\PYG@tc##1{\textcolor[rgb]{0.02,0.16,0.45}{##1}}}
\expandafter\def\csname PYG@tok@nv\endcsname{\def\PYG@tc##1{\textcolor[rgb]{0.73,0.38,0.84}{##1}}}
\expandafter\def\csname PYG@tok@s1\endcsname{\def\PYG@tc##1{\textcolor[rgb]{0.25,0.44,0.63}{##1}}}
\expandafter\def\csname PYG@tok@gp\endcsname{\let\PYG@bf=\textbf\def\PYG@tc##1{\textcolor[rgb]{0.78,0.36,0.04}{##1}}}
\expandafter\def\csname PYG@tok@sh\endcsname{\def\PYG@tc##1{\textcolor[rgb]{0.25,0.44,0.63}{##1}}}
\expandafter\def\csname PYG@tok@ow\endcsname{\let\PYG@bf=\textbf\def\PYG@tc##1{\textcolor[rgb]{0.00,0.44,0.13}{##1}}}
\expandafter\def\csname PYG@tok@sx\endcsname{\def\PYG@tc##1{\textcolor[rgb]{0.78,0.36,0.04}{##1}}}
\expandafter\def\csname PYG@tok@bp\endcsname{\def\PYG@tc##1{\textcolor[rgb]{0.00,0.44,0.13}{##1}}}
\expandafter\def\csname PYG@tok@c1\endcsname{\let\PYG@it=\textit\def\PYG@tc##1{\textcolor[rgb]{0.25,0.50,0.56}{##1}}}
\expandafter\def\csname PYG@tok@kc\endcsname{\let\PYG@bf=\textbf\def\PYG@tc##1{\textcolor[rgb]{0.00,0.44,0.13}{##1}}}
\expandafter\def\csname PYG@tok@c\endcsname{\let\PYG@it=\textit\def\PYG@tc##1{\textcolor[rgb]{0.25,0.50,0.56}{##1}}}
\expandafter\def\csname PYG@tok@mf\endcsname{\def\PYG@tc##1{\textcolor[rgb]{0.13,0.50,0.31}{##1}}}
\expandafter\def\csname PYG@tok@err\endcsname{\def\PYG@bc##1{\setlength{\fboxsep}{0pt}\fcolorbox[rgb]{1.00,0.00,0.00}{1,1,1}{\strut ##1}}}
\expandafter\def\csname PYG@tok@mb\endcsname{\def\PYG@tc##1{\textcolor[rgb]{0.13,0.50,0.31}{##1}}}
\expandafter\def\csname PYG@tok@ss\endcsname{\def\PYG@tc##1{\textcolor[rgb]{0.32,0.47,0.09}{##1}}}
\expandafter\def\csname PYG@tok@sr\endcsname{\def\PYG@tc##1{\textcolor[rgb]{0.14,0.33,0.53}{##1}}}
\expandafter\def\csname PYG@tok@mo\endcsname{\def\PYG@tc##1{\textcolor[rgb]{0.13,0.50,0.31}{##1}}}
\expandafter\def\csname PYG@tok@kd\endcsname{\let\PYG@bf=\textbf\def\PYG@tc##1{\textcolor[rgb]{0.00,0.44,0.13}{##1}}}
\expandafter\def\csname PYG@tok@mi\endcsname{\def\PYG@tc##1{\textcolor[rgb]{0.13,0.50,0.31}{##1}}}
\expandafter\def\csname PYG@tok@kn\endcsname{\let\PYG@bf=\textbf\def\PYG@tc##1{\textcolor[rgb]{0.00,0.44,0.13}{##1}}}
\expandafter\def\csname PYG@tok@o\endcsname{\def\PYG@tc##1{\textcolor[rgb]{0.40,0.40,0.40}{##1}}}
\expandafter\def\csname PYG@tok@kr\endcsname{\let\PYG@bf=\textbf\def\PYG@tc##1{\textcolor[rgb]{0.00,0.44,0.13}{##1}}}
\expandafter\def\csname PYG@tok@s\endcsname{\def\PYG@tc##1{\textcolor[rgb]{0.25,0.44,0.63}{##1}}}
\expandafter\def\csname PYG@tok@kp\endcsname{\def\PYG@tc##1{\textcolor[rgb]{0.00,0.44,0.13}{##1}}}
\expandafter\def\csname PYG@tok@w\endcsname{\def\PYG@tc##1{\textcolor[rgb]{0.73,0.73,0.73}{##1}}}
\expandafter\def\csname PYG@tok@kt\endcsname{\def\PYG@tc##1{\textcolor[rgb]{0.56,0.13,0.00}{##1}}}
\expandafter\def\csname PYG@tok@sc\endcsname{\def\PYG@tc##1{\textcolor[rgb]{0.25,0.44,0.63}{##1}}}
\expandafter\def\csname PYG@tok@sb\endcsname{\def\PYG@tc##1{\textcolor[rgb]{0.25,0.44,0.63}{##1}}}
\expandafter\def\csname PYG@tok@k\endcsname{\let\PYG@bf=\textbf\def\PYG@tc##1{\textcolor[rgb]{0.00,0.44,0.13}{##1}}}
\expandafter\def\csname PYG@tok@se\endcsname{\let\PYG@bf=\textbf\def\PYG@tc##1{\textcolor[rgb]{0.25,0.44,0.63}{##1}}}
\expandafter\def\csname PYG@tok@sd\endcsname{\let\PYG@it=\textit\def\PYG@tc##1{\textcolor[rgb]{0.25,0.44,0.63}{##1}}}

\def\PYGZbs{\char`\\}
\def\PYGZus{\char`\_}
\def\PYGZob{\char`\{}
\def\PYGZcb{\char`\}}
\def\PYGZca{\char`\^}
\def\PYGZam{\char`\&}
\def\PYGZlt{\char`\<}
\def\PYGZgt{\char`\>}
\def\PYGZsh{\char`\#}
\def\PYGZpc{\char`\%}
\def\PYGZdl{\char`\$}
\def\PYGZhy{\char`\-}
\def\PYGZsq{\char`\'}
\def\PYGZdq{\char`\"}
\def\PYGZti{\char`\~}
% for compatibility with earlier versions
\def\PYGZat{@}
\def\PYGZlb{[}
\def\PYGZrb{]}
\makeatother

\renewcommand\PYGZsq{\textquotesingle}

\begin{document}

\maketitle
\tableofcontents
\phantomsection\label{index::doc}



\section{Introduction}
\label{index:introduction}\label{index:openstreetmap-data-wrangling-with-mongodb}
The goal of this project is to use data wrangling (munging) techniques over data
obtained from the \href{http://www.openstreetmap.org}{OpenStreetMap}. The data
correspond to the area encompassing Santiago de Chile. The data wrangling
consisted in :
\begin{itemize}
\item {} 
Obtaining the data from an API in OSM format.

\item {} 
Parse the data using an python xml library.

\item {} 
Audit the data set and determine which problems we will solve

\item {} 
Once some functions to fix the chosen problems are created, a second parsing
is done from the OSM file, but this time fixing some of the issues.

\item {} 
Transform the parsed data into a valid JSON file to be imported by MongoDB

\item {} 
Import the JSON file into MongoDB

\end{itemize}


\section{Problems encountered in our map area}
\label{index:problems-encountered-in-our-map-area}

\subsection{Problems related to the parsing of the data}
\label{index:problems-related-to-the-parsing-of-the-data}
First we described some basic problems that need to be solved in order to convert
the XML file in a JSON file that can be imported by mongodb.
\begin{itemize}
\item {} 
Problematic characters: the fields names of the JSON document should include
only alphabet character. Also, according to the OSM wiki they should be avoided
in tags. We removed all \emph{tags} that were matched by the regular expression:

\begin{Verbatim}[commandchars=\\\{\}]
/[=\PYGZbs{}+/\PYGZam{}\PYGZlt{}\PYGZgt{};\PYGZbs{}\PYGZsq{}\PYGZdq{}{}`\PYGZbs{}?\PYGZpc{}\PYGZsh{}\PYGZdl{}@\PYGZbs{},\PYGZbs{}. \PYGZbs{}t\PYGZbs{}r\PYGZbs{}n]/
\end{Verbatim}

\item {} 
Tags with two sets of colons. Colons are present in the OSM's \emph{tags} to allow
the use of namespaces. It is not recommend to overuse this feature, so we
removed any tags with more than one namespace (i.e., more than one colon)

\end{itemize}

Both of these problems are solved using the appropriate algorithms that are
coded in the python files for this project.


\subsection{Problems related to the data audition}
\label{index:problems-related-to-the-data-audition}
Here a list of the problems we found within the data is presented. The will be
classified by the kind of problem defined in the Data Wrangling class.


\subsection{Validity}
\label{index:validity}\begin{itemize}
\item {} 
As in the Lesson 6 exercise, the street's name in the data set do not follow
a standard format when dealing with the 2 main kinds of streets used in Santiago's
street names: \emph{Avenida} and \emph{Pasaje}. In many cases they are abbreviated, and not
in a uniform way. \textbf{This problem was solved in the final parsing of the OSM file}

\item {} 
Many phone numbers do not follow a valid representation for Santiago de Chile.
The phone numbers should be in the format: \emph{+(Country\_Code) (Area\_Code) (Number)}.
\begin{itemize}
\item {} 
The \emph{Country\_Code} should be 56, and in many cases is not present, nor the plus
symbol at the beginning.

\item {} 
The \emph{Area\_Code} should be always 2 for residential lines, or 9X (with X a
number between 5 and 9) for cell phones.

\item {} 
The \emph{Number} should consist in an 8 digit number always starting with a 2
for residential lines; if a cell phone, the number should have 7 digits.

\end{itemize}

Only 129 phone numbers (of 993) are in the correct formatting.

\textbf{This problem is resolved by doing a query with pymongo, and then using the}
\textbf{{}`update{}` procedures}. The functions needed are at the end of the \emph{osm\_to\_json2.py} file.

\item {} 
E-mail address with no valid domain or plainly invalid (forbidden characters or
missing @). Examples:

\item {} 
Fields with multiple values without using semicolons to separate the different
items

\item {} 
Highways/streets with no `name' tag.

\item {} 
Address with multiple values for the street key.

\item {} 
Ways without any tags

\item {} 
Cross-field issues between \emph{address.street} and street \emph{name}: there are lots
of nodes with the field \emph{adress.street} where the street is not present in the
list of streets that we can retrieve from the highway {\color{red}\bfseries{}{}`}way{}`s.

\end{itemize}


\subsection{Completeness}
\label{index:completeness}\begin{itemize}
\item {} 
There are missing schools, traffic lights, hospitals, etc. A gold standard
database to fix completeness is not available.

\end{itemize}


\subsection{Consistency}
\label{index:consistency}\begin{itemize}
\item {} 
There are some nodes, very close in the same position, but with different tags.
For example, there is the case of a two nodes within a \emph{way} defining a school
area. However one of the has the tag \emph{amenity=school} and the other
\emph{amenity=fuel}: do we have a school or a gas station here? Is the \emph{area} a
school?

\end{itemize}


\section{Overview of the data}
\label{index:overview-of-the-data}

\subsection{File size:}
\label{index:file-size}
santiago.xlm            216 MB (OSM file)
santiago.xlm.json       237 MB (JSON file)


\subsection{Number of documents}
\label{index:number-of-documents}
\begin{Verbatim}[commandchars=\\\{\}]
\PYGZgt{} db.santiago.find().count()
1015506
\end{Verbatim}


\subsection{Number of nodes}
\label{index:number-of-nodes}
\begin{Verbatim}[commandchars=\\\{\}]
\PYGZgt{} db.santiago.find(\PYGZob{}type: \PYGZsq{}node\PYGZsq{}\PYGZcb{}).count()
814484
\end{Verbatim}


\subsection{Number of ways}
\label{index:number-of-ways}
\begin{Verbatim}[commandchars=\\\{\}]
\PYGZgt{} db.santiago.find(\PYGZob{}type: \PYGZsq{}way\PYGZsq{}\PYGZcb{}).count()
201022
\end{Verbatim}


\subsection{Number of unique users and top 10 users}
\label{index:number-of-unique-users-and-top-10-users}
\begin{Verbatim}[commandchars=\\\{\}]
\PYGZgt{} db.santiago.distinct(\PYGZdq{}created.user\PYGZdq{}).length
1003

\PYGZgt{} db.santiago.aggregate(
      [\PYGZob{}\PYGZdl{}group: \PYGZob{}\PYGZsq{}\PYGZus{}id\PYGZsq{}: \PYGZsq{}\PYGZdl{}created.user\PYGZsq{},
                 \PYGZsq{}count\PYGZsq{}: \PYGZob{}\PYGZdl{}sum:1\PYGZcb{}\PYGZcb{}\PYGZcb{},
       \PYGZob{}\PYGZdl{}sort: \PYGZob{}\PYGZsq{}count\PYGZsq{}: \PYGZhy{}1\PYGZcb{}\PYGZcb{},
       \PYGZob{}\PYGZdl{}limit: 10\PYGZcb{}])
 \PYGZob{} \PYGZdq{}\PYGZus{}id\PYGZdq{} : \PYGZdq{}Zambelli Limitada\PYGZdq{}, \PYGZdq{}count\PYGZdq{} : 244613 \PYGZcb{}
 \PYGZob{} \PYGZdq{}\PYGZus{}id\PYGZdq{} : \PYGZdq{}Fede Borgnia\PYGZdq{}, \PYGZdq{}count\PYGZdq{} : 204199 \PYGZcb{}
 \PYGZob{} \PYGZdq{}\PYGZus{}id\PYGZdq{} : \PYGZdq{}felipeedwards\PYGZdq{}, \PYGZdq{}count\PYGZdq{} : 107799 \PYGZcb{}
 \PYGZob{} \PYGZdq{}\PYGZus{}id\PYGZdq{} : \PYGZdq{}chesergio\PYGZdq{}, \PYGZdq{}count\PYGZdq{} : 61124 \PYGZcb{}
 \PYGZob{} \PYGZdq{}\PYGZus{}id\PYGZdq{} : \PYGZdq{}dintrans\PYGZus{}g\PYGZdq{}, \PYGZdq{}count\PYGZdq{} : 56281 \PYGZcb{}
 \PYGZob{} \PYGZdq{}\PYGZus{}id\PYGZdq{} : \PYGZdq{}madek\PYGZdq{}, \PYGZdq{}count\PYGZdq{} : 34397 \PYGZcb{}
 \PYGZob{} \PYGZdq{}\PYGZus{}id\PYGZdq{} : \PYGZdq{}ALE!\PYGZdq{}, \PYGZdq{}count\PYGZdq{} : 30174 \PYGZcb{}
 \PYGZob{} \PYGZdq{}\PYGZus{}id\PYGZdq{} : \PYGZdq{}toniello\PYGZdq{}, \PYGZdq{}count\PYGZdq{} : 27401 \PYGZcb{}
 \PYGZob{} \PYGZdq{}\PYGZus{}id\PYGZdq{} : \PYGZdq{}OttoPilot\PYGZdq{}, \PYGZdq{}count\PYGZdq{} : 16241 \PYGZcb{}
 \PYGZob{} \PYGZdq{}\PYGZus{}id\PYGZdq{} : \PYGZdq{}Chilestreet\PYGZdq{}, \PYGZdq{}count\PYGZdq{} : 15715 \PYGZcb{}
\end{Verbatim}


\section{Additional Ideas about the data set}
\label{index:additional-ideas-about-the-data-set}
There are several tools available to do a quality assurance over an OSM file
(JOSM/Validator) or online (Osmose, for selected areas). However they only deal
with validity, consistency and uniformity, while accuracy and completeness are
left out.

This is understandable, since the accuracy and completeness depend on having
\emph{gold standard} databases that we can use to perform these steps in the measure
of the data quality. A great project would be to help the local city government
to update and maintain the gold standard databases.

Currently, in Chile there is \href{http://datos.gob.cl}{web page} associated to the
National Government that aims to provide this information to the users that
require it. The information can't only be queried through the web interface, but
an API has also been implemented. However many problems remain:
\begin{itemize}
\item {} 
Data is outdated and no longer maintained: besides some small number of
databases, like the database with all the public transportation information
for Santiago and the database with all the schools in the Santiago area, most
of the geolocation data sets are outdated.

\item {} 
Several geolocation data sets are missing: a database of the streets of Santiago
with valid names, a database related to hospitals and clinics, a database with
police stations, etc. They might actually exist, but they are not publicly
available in this web page nor through the API.

\end{itemize}

Another good project would consist in resume the work at the
\href{http://wiki.openstreetmap.org/wiki/WikiProject\_Chile}{WikiProject Chile}
OpenStreetMap page. There is some work already done there, which aims mainly to
use a consistent set of tags and values to give consistency and uniformity to
chilean map features, but is still clearly incomplete and kind of dead, since
no modifications have been done in the last year.


\subsection{Additional data exploration using MongoDB queries}
\label{index:additional-data-exploration-using-mongodb-queries}

\subsubsection{Number of schools in Santiago}
\label{index:number-of-schools-in-santiago}
\begin{Verbatim}[commandchars=\\\{\}]
\PYGZgt{} db.santiago.find(\PYGZob{}\PYGZsq{}amenity\PYGZsq{}:\PYGZsq{}school\PYGZsq{}\PYGZcb{}).count()
2550
\end{Verbatim}


\subsubsection{Number of police stations in Santiago}
\label{index:number-of-police-stations-in-santiago}
\begin{Verbatim}[commandchars=\\\{\}]
\PYGZgt{} db.santiago.find(\PYGZob{}\PYGZsq{}amenity\PYGZsq{}:\PYGZsq{}police\PYGZsq{}\PYGZcb{}).count()
72
\end{Verbatim}


\subsubsection{Number of highways (streets) without name}
\label{index:number-of-highways-streets-without-name}
\begin{Verbatim}[commandchars=\\\{\}]
\PYGZgt{} db.santiago.find(\PYGZob{}highway:\PYGZob{}\PYGZdl{}exists:1\PYGZcb{}, type:\PYGZsq{}way\PYGZsq{}, name:\PYGZsq{}\PYGZsq{}\PYGZcb{}).count()
47
\end{Verbatim}


\subsubsection{The 10 districts with the higher number of bus stops}
\label{index:the-10-districts-with-the-higher-number-of-bus-stops}
\begin{Verbatim}[commandchars=\\\{\}]
\PYGZgt{} db.santiago.aggregate(
    \PYGZob{}\PYGZdl{}match: \PYGZob{}type: \PYGZsq{}node\PYGZsq{}, highway:\PYGZsq{}bus\PYGZus{}stop\PYGZsq{}\PYGZcb{}\PYGZcb{},
    \PYGZob{}\PYGZdl{}group: \PYGZob{}\PYGZsq{}\PYGZus{}id\PYGZsq{}: \PYGZsq{}\PYGZdl{}is\PYGZus{}in:city\PYGZsq{}, \PYGZsq{}count\PYGZsq{}: \PYGZob{}\PYGZdl{}sum:1\PYGZcb{}\PYGZcb{}\PYGZcb{},
    \PYGZob{}\PYGZdl{}sort: \PYGZob{}\PYGZsq{}count\PYGZsq{}:\PYGZhy{}1\PYGZcb{}\PYGZcb{},
    \PYGZob{}\PYGZdl{}limit:10\PYGZcb{})

 \PYGZob{} \PYGZdq{}\PYGZus{}id\PYGZdq{} : \PYGZdq{}Puente Alto\PYGZdq{}, \PYGZdq{}count\PYGZdq{} : 800 \PYGZcb{}
 \PYGZob{} \PYGZdq{}\PYGZus{}id\PYGZdq{} : \PYGZdq{}Maipú\PYGZdq{}, \PYGZdq{}count\PYGZdq{} : 746 \PYGZcb{}
 \PYGZob{} \PYGZdq{}\PYGZus{}id\PYGZdq{} : \PYGZdq{}La Florida\PYGZdq{}, \PYGZdq{}count\PYGZdq{} : 668 \PYGZcb{}
 \PYGZob{} \PYGZdq{}\PYGZus{}id\PYGZdq{} : \PYGZdq{}Santiago\PYGZdq{}, \PYGZdq{}count\PYGZdq{} : 620 \PYGZcb{}
 \PYGZob{} \PYGZdq{}\PYGZus{}id\PYGZdq{} : \PYGZdq{}San Bernardo\PYGZdq{}, \PYGZdq{}count\PYGZdq{} : 476 \PYGZcb{}
 \PYGZob{} \PYGZdq{}\PYGZus{}id\PYGZdq{} : \PYGZdq{}Pudahuel\PYGZdq{}, \PYGZdq{}count\PYGZdq{} : 416 \PYGZcb{}
 \PYGZob{} \PYGZdq{}\PYGZus{}id\PYGZdq{} : \PYGZdq{}Las Condes\PYGZdq{}, \PYGZdq{}count\PYGZdq{} : 408 \PYGZcb{}
 \PYGZob{} \PYGZdq{}\PYGZus{}id\PYGZdq{} : \PYGZdq{}Ñuñoa\PYGZdq{}, \PYGZdq{}count\PYGZdq{} : 362 \PYGZcb{}
 \PYGZob{} \PYGZdq{}\PYGZus{}id\PYGZdq{} : \PYGZdq{}La Pintana\PYGZdq{}, \PYGZdq{}count\PYGZdq{} : 353 \PYGZcb{}
 \PYGZob{} \PYGZdq{}\PYGZus{}id\PYGZdq{} : \PYGZdq{}Peñalolén\PYGZdq{}, \PYGZdq{}count\PYGZdq{} : 352 \PYGZcb{}
\end{Verbatim}



\renewcommand{\indexname}{Index}
\printindex
\end{document}
